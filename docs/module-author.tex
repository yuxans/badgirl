\documentclass{article}[12pt]

\title{Moobot Module Author Guide}
\author{Daniel DiPaolo}
%%%%%%%%%%%%%% DOCUMENT BEGINS HERE %%%%%%%%%%%%%%%%%%
\begin{document}
\maketitle
\newpage

\tableofcontents
\newpage

\section{Introduction}
This guide gives a brief introduction on the guidelines for writing a module
for Moobot as well as offering some tips.
\subsection{Prerequisites}
You more or less have to have a working Moobot installation to test out
modules.  You technically can write them without having a Moobot running, but
testing is not as simple that way.  Plus, since Moobot modules are dynamically
reloadable, once you start working on your module you can simply reload it
every time it changes instead of restarting the bot.

That said, you obviously need Python working, and you need the
moobot\_module.py file to import the base module class.  It is included with
the moobot installation as a standard.

\section{Example: the ``hi'' module}
This is quite simply the dumbest Moobot module of the whole bunch, but it
serves its purpose in demonstrating the basic requirements of Moobot modules.
The contents of \texttt{\textbf{hi.py}} can be seen in figure
\ref{fig:hi-listing}.

\begin{figure}[hp]
	\begin{center}
	\begin{verbatim}
#!/usr/bin/env python

"""hi.py - dummy function that says Hi """

from moobot_module import MooBotModule

class hi(MooBotModule):
	def __init__(self):
		self.regex="^hi$"

	def handler(self, **args):
		"""A dummy handler we used for testing -- this is the first handler
		we wrote"""
		from irclib import Event
		return Event("privmsg", "", self.return_to_sender(args), [ "howdy" ])
	\end{verbatim}
	\end{center}
	\caption{Listing of \texttt{hi.py}}
	\label{fig:hi-listing}
\end{figure}

Now let's look at each of the components.
\subsection{File docstring}
This isn't required for modules, but it is always nice to use even if it isn't
very wordy or descriptive.  We have one here just for the sake of encouraging
the usage of document docstrings.

\subsection{\texttt{import} statements}
Though you don't necessarily have to use \texttt{from moobot\_module import
MooBotModule}, you do have to include \texttt{MooBotModule} somehow.  This is
the class that all modules inherit from.  It contains some nice functions that
we will discuss later.  For now all you need to know is that you have to have
it.

\subsection{Class definition(s)}
The class definition(s) don't have to match the filename, but if you are going
to make a single module, the filename and the class name should both describe
that module (briefly) and you probably want to name them the same, like in our
example here.

However, note that you do have to derive your class from the MooBotModule base
class.  Right now there is no specific type-checking making sure that this is
the case (so, technically, as long as your module conforms to the appropriate
structure discussed below and shown in the base module class), but please be
nice and use the base module class.

\subsection{\texttt{\_\_init\_\_}}
All this function has to do right now is set \texttt{self.regex} to the
regular expression you wish your Moobot to use this module for.  We only want
our Moobot to respond to someone that says "hi" and nothing else, so we set
our regular expression to "\^hi\$", meaning it must start and end with ``hi''
(note: this does not include the bot name).  Any Python regular expression may
be used for this, as it uses the built-in \texttt{re} module that comes with
Python.

\subsection{\texttt{handler}}
This is the main component of all modules, and this is the function that gets
called when the module is found within the list of handlers registered with
Moobot (discussed in the next section).  The arguments for this function
\emph{must} be as shown in the example: \texttt{(self, **args)}.  Of course,
the \texttt{self} refers to the explicit scoping that Python uses within its
class methods.  The \texttt{**args} refers to the arguments that the module
gets passed from Moobot during operation.  Presently, these consist of:

\begin{itemize}
	\item[\texttt{args["source"]}]	Full nick and hostmask of the person that
	said whatever item that Moobot is processing.  Example:
	\texttt{moobot!moobot@moobot.sourceforge.net}
	\item[\texttt{args["text"]}]	Full text that was said to cause this
	module to be activated.  This \emph{always} includes the bot's nickname
	even if the shorthand form of addressing is used to address him.  The
	bot's name is always pre-pended like - "moobot: "
	\item[\texttt{args["type"]}]	Either ``pubmsg'' or ``privmsg''.  It is
	``pubmsg'' if the message was read in a channel that the Moobot is in, and
	it is ``privmsg'' if the message was sent to Moobot directly.
	\item[\texttt{args["channel"]}]	Set to the channel that the message was
	set in if the message was a public message (a ``pubmsg''), and probably
	set to ``'' for private messages (``privmsg''s), but I'm too lazy to check
	it out - feel free to send a patch for this document correcting this.
\end{itemize}

All \texttt{handler()} functions are expected to return \texttt{Event}s, which
can be imported from \texttt{irclib}.

\subsubsection{\texttt{Event} description}
There are four components to an \texttt{Event}.  The first component is the
\emph{type}, which is basically always either a ``privmsg'' or an ``action'',
corresponding to whether the event is to be spoken or ``acted out'' (as much
as things can be acted out on IRC).  The second component is the \emph{source},
which might as well always be set to ``'', as the event handler code within
Moobot ignores it and sets it to the appropriate nickmask for your Moobot.
The third argument is the \emph{target} for the Event.  This should be set to
either a channel or a nick.  This is usually set by the output of the
\texttt{return\_to\_sender} function described in section
\ref{sec:ret-to-send}, though it can be set by any means you wish.  The last
component is the actual \emph{message}.  Actually it is a list of messages
which will be said (or performed, for actions) in sequence.  Usually modules
are encouraged to just output one line to the channel to avoid flooding.

In our example, we return a ``privmsg'' (which is somewhat misleading because
messages to a channel are ``privmsg''s also), to either the person who said it
to us or to the channel it was said in, and we simply say ``howdy''.  Fairly
simple, no?

\subsection{Other functions}
Of course, if it makes sense for your module you may include other functions
within your module classes if you like, but make sure you follow the Python
syntax convention of explicit scoping and refer to all other class methods
using the explicit \texttt{self} argument.

\subsection{Functions built in to all Moobot modules}
\subsubsection{\texttt{return\_to\_sender}}
\label{sec:ret-to-send}
This is a very useful function (so useful, we included it in the base module,
imagine that), and it can be seen in the listing of \texttt{hi.py} in figure
\ref{fig:hi-listing}.  This function basically takes a dict (which, for
pretty much all intents and purposes should be \texttt{args} most of
the time).  If the message is a private message, returns the nick of the
person who said the message.  Otherwise, it returns a string with the name of
the channel that the message was said in.  This is pretty much almost always
the behavior desired - answer messages said in the channel by replying in the
channel, and answer private messages with private messages in turn.

\section{How Moobot uses modules}
Modules are registered with the Moobot object before the \texttt{start()}
method of the Moobot is called.  To register a module with Moobot, you simply
use the \texttt{add\_handler()} method of your initialized Moobot object.  The
lines used to add the \texttt{hi} module are shown in figure \ref{fig:hi-add}.

\begin{figure}[hp]
	\begin{center}
		\begin{verbatim}
		import hi
		bot.add_handler(hi, "hi")
		\end{verbatim}	
	\end{center}
	\caption{\texttt{hi} module inclusion lines}
	\label{fig:hi-add}
\end{figure}

This code assumes that the initialized Moobot object is \texttt{bot}.  The
first argument to \texttt{add\_handler()} is the module instance, \texttt{hi}.
So it is important to import the module first before trying to add Moobot
modules (the classes within your module file) with the \texttt{add\_handler()}
function.  The second argument is the name of the class that contains the code
for a module and the regex it looks for.

The order of modules is the order in which the regular expressions are
checked.
\end{document}

